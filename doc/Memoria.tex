%%%%%%%%%%%%%%%%%%%%%%%%%%%%%%%%%%%%%%%%%%%%%%%%%%%%%%%%%%%%%%%%%%%%%%%%%%%%%%%%
% Memoria para trabajos y entregas de laboratorio de la Escuela Superior de    %
% Informática (ESI) de Ciudad Real, UCLM.                                      %
%   Versión: Octubre - 2018                                                    %
%   Desarrollada por José Ángel Martín Baos                                    %
%                                                                              %
% Recursos:			                                                           %
%   - Contenidos del curso “LaTeX esencial para preparación de Trabajo Fin de  %
%     Grado, Tesis y otros documentos académicos” impartido por el profesor    %
%     Jesús Salido.                                                            %
%                                                                              %
% Disponible en: https://github.com/JoseAngelMartinB/PlantillaTrabajosLaTeX    %
%%%%%%%%%%%%%%%%%%%%%%%%%%%%%%%%%%%%%%%%%%%%%%%%%%%%%%%%%%%%%%%%%%%%%%%%%%%%%%%%

\documentclass[11pt]{article}

% PAQUETES USADOS:
\usepackage{natbib}
\usepackage{url}
\usepackage[utf8]{inputenc} % Codificación (Permite carácteres españoles)
\usepackage{amsmath}
\usepackage{graphicx}
\graphicspath{{images/}} % Carpeta en la cual se van a buscar las imagenes
\usepackage{subfigure}	% Permite la Inclusión de subfiguras
%\usepackage{parskip} % Suprime la identación de los parrafos.
\setlength{\parskip}{3mm} % Longitud del espaciado entre parrafos
\usepackage[hidelinks]{hyperref} % Referencias (links)
\usepackage{fancyhdr}
\usepackage{vmargin}
\usepackage{paralist} % Permite un mayor control sobre las listas
\usepackage{textcomp,marvosym,pifont} % Generación de símbolos especiales
\usepackage[usenames,dvipsnames,svgnames,x11names,table]{xcolor}

\usepackage{listings}
\usepackage{xcolor}
\usepackage{url}
\usepackage{booktabs}

% CONFIGURACIÓN DE LA PÁGINA:
\setpapersize{A4} % Formato del papel - A4
\setmarginsrb{3 cm}{2.5 cm}{3 cm}{2.5 cm}{1 cm}{1.5 cm}{1 cm}{1.5 cm} % Margenes

% Complemento para insertar código en la memoria:
%   Basado en 'Listados de código cómodos y resultones con listings'
%   de David Villa en http://crysol.org/es/node/909
\usepackage{color}
\definecolor{gray97}{gray}{.97}
\definecolor{gray75}{gray}{.75}
\definecolor{gray45}{gray}{.45}
\usepackage{listings}
\lstset{ frame=Ltb,
	framerule=0pt,
	aboveskip=0.5cm,
	framextopmargin=3pt,
	framexbottommargin=3pt,
	framexleftmargin=0.4cm,
	framesep=0pt,
	rulesep=.4pt,
	backgroundcolor=\color{gray97},
	rulesepcolor=\color{black},
	texcl=true,
	%
	stringstyle=\ttfamily,
	showstringspaces = false,
	basicstyle=\small\ttfamily,
	commentstyle=\color{gray45},
	keywordstyle=\bfseries,
	%
	numbers=left,
	numbersep=15pt,
	numberstyle=\tiny,
	numberfirstline = false,
	breaklines=true,
}
% Minimizar fragmentado de listados
\lstnewenvironment{listing}[1][]
{\lstset{#1}\pagebreak[0]}{\pagebreak[0]}

\lstdefinestyle{consola}
{basicstyle=\scriptsize\bf\ttfamily,
	backgroundcolor=\color{gray75},
}

\lstdefinestyle{C}
{language=C,
}

%OTROS PAQUETES:
\usepackage{float} % Permite usar H en las figuras, de manera que se coloquen en la posición exacta en la que están en el código.

% Añade un comando para crear indicaciones de pulsación de teclas
\usepackage{tikz} % Paquete especializado en gráficos
\usetikzlibrary{shadows} % Necesario para poder crear nuevo comando de indicación de pulsación de tecla.
\newcommand*\tecla[1]{%
	\tikz[baseline=(key.base)]
	\node[%
	draw,
	fill=white,
	drop shadow={shadow xshift=0.25ex,shadow yshift=-0.25ex,fill=black,opacity=0.75},
	rectangle,
	rounded corners=2pt,
	inner sep=1pt,
	line width=0.5pt,
	font=\scriptsize\sffamily
	](key) {#1\strut}
	;
}

\newif\ifspanish % Condicional que permite seleccionar el lenguage.
\newif\ifmultipleauthors % Condicional que permite multiples autores
\spanishtrue
\multipleauthorsfalse


%%%%%%%%%%%%%%%%%%%%%%%%%%%%%%%%%%%%%%%%%%%%%%%%%%%%%%%%%%%%%%%%%%%%%%%%%%%%%%%%
%%%%%%%%%				Principales variables del documento			   %%%%%%%%%

\title{Filtros avanzados con \textit{MPI} y \textit{Python}}							% Titulo
\author{Mario Pérez Sánchez-Montañez}							% Autor
%\author{Nombre~Apellido \\ \texttt{email.alumno@alu.uclm.es}} % Autor con email
%\author{José~Ángel~Martín~Baos \\ Autor~2 \\ Autor~3}	% Multiples autores
\date{\today}											% Fecha
\newcommand{\subject}{Diseño de Infraestructura de Red}						% Asignatura
\newcommand{\course}{Grado en Ingeniería Informática \\ 3º - Ingeniería de Computadores}	% Curso
%\newcommand{\course}{Máster Universitario en Ingeniería Informática}	% Curso
\newcommand{\courseyear}{2019 -- 2020} 					% Curso académico
%\spanishfalse	    	% Descomentar esta línea si el trabajo está en inglés
%\multipleauthorstrue   % Descomentar esta línea si hay varios autores

%%%%%%%%%%%%%%%%%%%%%%%%%%%%%%%%%%%%%%%%%%%%%%%%%%%%%%%%%%%%%%%%%%%%%%%%%%%%%%%%

\ifspanish
	\usepackage[spanish]{babel} % Paquete de español
	\newcommand{\dateText}{Fecha:}
	\renewcommand{\lstlistingname}{Listado} % Renombrar listados para que aparezcan en español.
	% Algoritmos
	\usepackage[ruled,vlined,spanish]{algorithm2e} % Permite pseudocódigos. NECESARIO INSTALAR texlive-science (sudo apt-get install texlive-science)
\else
	\usepackage[english]{babel} % Paquete de inglés
	\newcommand{\dateText}{Date:}
	% Algoritmos
	\usepackage[ruled,vlined,english]{algorithm2e}
\fi

\makeatletter
\let\thetitle\@title
\let\theauthor\@author
\let\thedate\@date
\makeatother

% Formato de página:
\pagestyle{fancy}		% Formato por defecto - Recomendado
%\pagestyle{headings} 	% Formato para libros
\fancyhf{}
\ifmultipleauthors
	\chead{\thetitle}
\else
	\rhead{\theauthor}
	\lhead{\thetitle}
\fi
\cfoot{\thepage}

\begin{document}

%%%%%%%%%%%%%%%%%%%%%%%%%%%%%%%%%%%%%%%%%%%%%%%%%%%%%%%%%%%%%%%%%%%%%%%%%%%%%%%%
%%%%%%%%							Portada							   %%%%%%%%%

\begin{titlepage}
	\centering
	\begin{minipage}[t]{\textwidth}
		\raisebox{-0.5\height}{\includegraphics[scale = 0.8]{UCLM_Logo.pdf}} 	% Logo de la universidad
		\hspace{\fill}
		\raisebox{-0.5\height}{\includegraphics[scale = 0.24]{esi.pdf}}
	\end{minipage}
	\\[2.25 cm]
    \textsc{\LARGE Universidad de Castilla-La Mancha}\\[0.5 cm]	% Nombre de la universidad
    \textsc{\LARGE Escuela Superior de Informática}\\[2.0 cm]
	\textsc{\Large \subject}\\[0.5 cm]				% Asignatura
	\textsc{\large \course \\ \courseyear}\\[2 cm]				% Curso
	\rule{\linewidth}{0.2 mm} \\[0.4 cm]
	{ \huge \bfseries \thetitle}\\
	\rule{\linewidth}{0.2 mm} \\[2.5 cm]

	\vspace*{\fill}
	\begin{minipage}{0.4\textwidth}
		\begin{flushleft} \large
			\ifspanish
				\ifmultipleauthors
					\emph{Autores:}\\
				\else
					\emph{Autor:}\\
				\fi
			\else
				\ifmultipleauthors
					\emph{Authors:}\\
				\else
					\emph{Author:}\\
				\fi
			\fi
			\theauthor
			\end{flushleft}
			\end{minipage}~
			\begin{minipage}{0.4\textwidth}
			\begin{flushright} \large
			\emph{\dateText} \\
			\thedate
		\end{flushright}
	\end{minipage}\\[2.25 cm]


\end{titlepage}

%%%%%%%%%%%%%%%%%%%%%%%%%%%%%%%%%%%%%%%%%%%%%%%%%%%%%%%%%%%%%%%%%%%%%%%%%%%%%%%%
%%%%%%%%%						   	Índice					      	   %%%%%%%%%

\tableofcontents
\pagebreak


%%%%%%%%%%%%%%%%%%%%%%%%%%%%%%%%%%%%%%%%%%%%%%%%%%%%%%%%%%%%%%%%%%%%%%%%%%%%%%%%
%%%%%%%%%							Documento						   %%%%%%%%%

%\section{Enunciado}
%Dado un archivo con nombre datos.dat, cuyo contenido es una lista de valores
%separados por comas, nuestro programa realizará lo siguiente:
%El proceso de rank 0 destribuirá a cada uno de los nodos de un Hipercubo de
%dimensión D, los $2^D$ numeros reales que estarán contenidos en el archivo
%datos.dat. En caso de que no se hayan lanzado suficientes elementos de proceso
%para los datos del programa, éste emitirá un error y todos los procesos
%finalizarán.
%En caso de que todos los procesos han recibido su correspondiente elemento,
%comenzará el proceso normal del programa.
%Se pide calcular el elemento mayor de toda la red, el elemento de proceso con
%rank 0 mostrará en su salida estándar el valor obtenido. La complejidad del
%algoritmo no superará O($\log_{2}(n)$) Con n número de elementos de la
%red.
%
%\section{Planteamiento de la solución}
%Las redes de hipercubos son un tipo de topología de red utilizada para conectar múltiples procesadores. Las redes de hipercubos constan de $2^m$ nodos, que forman los vértices de los cuadrados para crear una conexión entre redes. Un hipercubo es básicamente una red de malla multidimensional con dos nodos en cada dimensión, como se puede ver en la Figura \ref{hipercubo}.
%
%\begin{figure}[!h]
%	\centering
%	\includegraphics[width=10cm]{hipercubo.png}
%	\caption{Ejemplo topología hipercubo \cite{imagen_hipercubo}}
%	\label{hipercubo}
%\end{figure}
%
%La solución planteada se basa en crear \textit{$2^D$} procesos (siendo \textit{D} la dimensión del hipercubo). En este caso planteo una solución de dimensión 3, por lo tanto, 8 procesos. En el fichero \textit{\textbf{hypercube.c}} se especifica una constante para la dimensión. Por ello, si se quiere cambiar la dimensión del hipercubo se debe modificar esta constante y los nodos indicados en el \textit{\textbf{Makefile}}.
%
%El el proceso con rank 0 se encarga de leer el fichero \textit{\textbf{datos.dat}} y de enviar el respectivo número a cada nodo, incluido sí mismo. Si este proceso se realiza con éxito los procesos comenzarán a enviar sus número a sus vecinos en busca del máximo número de la red, en caso contrario la variable \textit{\textbf{end}} se activará todos los procesos finalizarán.
%
%
%\subsection{Cálculo de vecinos}
%\label{vecinos}
%La parte fundamental de la solución es el cálculo de los respectivos vecinos, ya que esto especificará la topología de la red \cite{topologia_hipercubo}. Cada nodo tiene dos vecinos, que se calculan aplicando la operación binaria \textit{\textbf{XOR}} \cite{xor} (que produce verdadero solo cuando las entradas difieren), aplicada según este algoritmo \cite{algoritmo} (ver Figura \ref{patron})
%
%\begin{figure}[!h]
%	\centering
%	\includegraphics[width=10cm]{vecinos.png}
%	\caption{Patrón de comunicación hipercubo \cite{imagen_algoritmo}}
%	\label{patron}
%\end{figure}
%
%
%
%\section{Diseño de la solución}
%\subsection{Estructura de la función principal}
%En esta función se inicializan las variables principales de nuestro programa y se realizan las llamadas al resto de funciones, según la acción que se deba realizar. En cuanto a \textit{MPI} se realizan las siguientes acciones:
%
%\begin{itemize}
%	\item Inicializar la estructura de comunicación de MPI entre los procesos \cite{mpi_init}.
%	\item Determinar el \textit{rank} (identificador) del proceso que lo llama dentro del comunicador seleccionado \cite{mpi_rank}.
%	\item Determinar el tamaño del comunicador seleccionado, es decir, el número de procesos que están actualmente asociados a este \cite{mpi_size}.
%\end{itemize}
%
%
%\begin{lstlisting}
%int main(int argc, char *argv[]){
%	double *data = malloc(N_NODES * sizeof(double));
%	int length;
%	int rank, size;
%	double num;
%	
%	MPI_Init(&argc, &argv);
%	MPI_Comm_rank(MPI_COMM_WORLD,&rank);
%	MPI_Comm_size(MPI_COMM_WORLD,&size);
%
%	if (rank == 0){
%		get_data(data, &length);
%		
%		check_data(length, LENGTH_MSG);
%		
%		if (!end) check_data(size, SIZE_MSG);
%		
%		if (!end) send_data(data);
%	}
%
%	/* Get confirmation to continue from first node */
%	MPI_Bcast(&end,1,MPI_INT,0,MPI_COMM_WORLD);
%	
%	if(!end){
%		/* Wait the number */
%		MPI_Recv(&num, 1, MPI_DOUBLE, 0, MPI_ANY_TAG, MPI_COMM_WORLD, NULL);
%		calculate_max(rank, num);
%	}
%	
%	MPI_Finalize();
%	
%	return EXIT_SUCCESS;
%}
%\end{lstlisting}
%
%A continuación, se van a describir las distintas acciones que se llevan a cabo en el código anterior con las respectivas funciones utilizadas.
%
%\subsection{Lectura del fichero}
%La lectura del fichero para obtener los distintos números se realiza con se realiza con la siguiente función:
%
%\begin{lstlisting}
%void get_data(double *data, int *length){
%	/* For load data from datos.dat */
%	
%	FILE *file;
%	char *aux = malloc(1024 * sizeof(char));
%	char *n;
%	
%	if ((file = fopen(FILE_NAME, "r")) == NULL){
%		fprintf(stderr, "Error opening file\n");
%		end = TRUE; 
%	}else{
%		*length = 0;
%		
%		fscanf(file, "%s", aux);
%		fclose(file);
%		
%		data[(*length)++] = atof(strtok(aux,","));
%		
%		while((n = strtok(NULL, ",")) != NULL)
%			data[(*length)++] = atof(n);
%	}
%	
%	free(aux);
%
%}
%\end{lstlisting}
%
%Los parámetros de la función son un array de \textit{double} para almacenar los números leídos del fichero y una variable entera que representa la cantidad de números obtenidos. Ambos parámetros se introducen por referencia, ya que las modificaciones las necesitamos posteriormente.
%
%\subsection{Tratamiento de errores}
%Una vez concluida la carga de los datos, se realizarán dos comprobaciones para iniciar los cálculos.
%
%\begin{itemize}
%	\item En primer lugar, se comprueba el tamaño obtenido de MPI (variable \textit{\textbf{size}})
%	\item En segundo lugar, si la comprobación anterior ha resultado exitosa se comprueba el número de elementos obtenidos, almacenado en la variable \textit{\textbf{length}}.
%\end{itemize}
%
%Ambas comprobaciones se realizan con la siguiente función, ya que las dos valores a comprobar se comparan con el número de nodos totales (\textit{\textbf{$2^D$}}).
%
%\begin{lstlisting}
%void check_data(int var, char *type){
%	/* For check length or size */
%	
%	if (var != N_NODES){
%		fprintf(stderr, "Error in data %s\n", type);
%		end = TRUE;
%	}
%}
%\end{lstlisting}
%
%\subsection{Envío de datos}
%
%Las comprobaciones activarán la variable global \textit{\textbf{end}} si hay algún valor incorrecto. Si los valores comprobados son correctos se procede a enviar los datos, utilizando la siguiente función.
%
%\begin{lstlisting}
%void send_data(double *data){
%	/* Send numbers to all the nodes */
%	
%	double buff_num;
%	int i;
%	
%	for(i=0; i < N_NODES; i++){
%		buff_num = data[i];
%		MPI_Send(&buff_num, 1, MPI_DOUBLE, i, 0, MPI_COMM_WORLD);
%		printf("%.2f sended to %d node\n",buff_num, i);
%	}
%	
%	free(data);
%}
%
%\end{lstlisting}
%
%En lo relativo a MPI, se hace uso de la función de envío  como se puede observar en la línea 9 del bloque de código anterior.
%
%\subsection{Recepción de datos}
%Antes de realizar la recepción de los datos se realiza un envío de la variable de comprobación (\textit{\textbf{end}}) a todos los procesos utilizando la función de \textit{broadcast} de MPI \cite{mpi_bcast}. La razón es que los diferentes procesos no deben recibir sus datos si las comprobaciones no han sido exitosas. Para la recepción del número que le corresponde a cada nodo se ha utilizado la primitiva básica de recepción de MPI \cite{mpi_recv}.
%
%\subsection{Cálculo del máximo número de la red}
%El objetivo del programa es que todos los nodos de la red se queden con el número máximo de todo el conjunto de nodos. Para ello, los nodos se deben enviar los distintos números de unos a otros hasta encontrar el máximo. Tras esto, el primer nodo (con \textit{rank} 0) imprimirá el número que tiene actualmente, es decir, el mínimo de toda la red.
%
%\subsubsection{Implementación de los vecinos}
%Los distintos nodos de la red enviarán y recibirán los números de sus vecinos respectivamente. Es por ello, que lo más importante para que el cálculo del máximo de la red se realice correctamente es calcular los cuatro vecinos que tiene cada nodo de forma correcta. 
%
%El siguiente bloque de código representa la implementación del cálculo de los vecinos para un determinado nodo de la red. La forma de obtener los vecinos se ha explicado de manera teórica en la sección \ref{vecinos}.
%
%\begin{lstlisting}
%void hypercube_neighbors(int rank, int neighbors[]){
%	/* Calculate the neighbors */
%	
%	int i;
%	
%	for(i = 0; i < D; i++)
%		neighbors[i] = XOR(rank, (int)pow(2,i));
%}
%\end{lstlisting}
%
%En la topología hipercubo, cada elemento de procesamiento itera ejecutando la expresión $XOR(rank, 2^i)$ para obtener cada vecino. Siendo $i$ la variable del bucle que toma valores desde 0 hasta \textit{D-1}. Por ejemplo, para 3 dimensiones cada nodo tendrá tres vecinos.
%
%La función tiene como parámetro un array, donde se almacenarán los \textit{ranks} de los vecinos del nodo que haya ejecutado la función. Como los vectores en C se pasan por referencia, los valores de este vector los podremos utilizar después de ejecutar está función.
%
%\subsubsection{Envío y recepción de los números}
%Una vez calculados los vecinos de cada nodo, los procesos comenzarán a enviar su número a sus vecinos y recibir otros valores para comparar el que se recibe con el que se tiene actualmente.
%
%\begin{lstlisting}
%void calculate_max(int rank, double num){
%	/* Calculate the maximum number of all the nodes */
%	
%	int neighbors[D];
%	double his_num;
%	int i;
%	
%	hypercube_neighbors(rank, neighbors);
%	
%	for(i=1; i < D; i++){
%		MPI_Send(&num, 1, MPI_DOUBLE, neighbors[i], 1, MPI_COMM_WORLD);
%		MPI_Recv(&his_num, 1, MPI_DOUBLE, neighbors[i], 1,MPI_COMM_WORLD, NULL);
%		num = MAX(num, his_num);
%	}
%	
%	if(rank == 0) printf("\nThe maximum number is: %.2f\n",num);
%
%}
%\end{lstlisting}
%
%El algoritmo para calcular el máximo de la red hipercubo primero calcula todos los vecinos utilizando la función explicada anteriormente y se realiza el intercambio entre vecinos utilizando las funciones de MPI de envío \cite{mpi_send} y recepción \cite{mpi_recv}.
%
%\subsubsection{Implementación final}
%Tras un estudio de la práctica y basándome en \cite{algoritmo} he realizado un nueva implementación. En la cual se realiza el cálculo del vecino correspondiente y el intercambio de números en el mismo bucle, como se puede ver en el siguiente bloque de código:
%\begin{lstlisting}
%void calculate_max(int rank, double num){
%	/* Calculate the maximum number of all the nodes 
%	
%	double his_num;
%	int neighbor, i;
%	
%	for(i=0; i < D; i++){
%		neighbor = XOR(rank, (int)pow(2,i));
%		MPI_Send(&num, 1, MPI_DOUBLE, neighbor, 1, MPI_COMM_WORLD);
%		MPI_Recv(&his_num, 1, MPI_DOUBLE, neighbor, 1,MPI_COMM_WORLD, NULL);
%		num = MAX(num, his_num);
%	}
%	
%	if(rank == 0) printf("\nThe maximum number is: %.2f\n",num);
%
%}
%\end{lstlisting}
%
%Se puede observar como se calcula el vecino correspondiente a la iteración del bucle en la que se encuentre el programa y, a continuación, se realiza el intercambio de números buscando el mayor.


% Sección: Intruducción
%\section{Introducción}
%Este es un ejemplo de una plantilla hecha con \LaTeX{} para la realización de trabajos y entregas de laboratorio de la ESI (Escuela Superior de Informática) de Ciudad Real, Universidad de Castilla-La Mancha.
%
%En la plantilla se han concentrado todas las características de \LaTeX{} que pueden ser requeridas cuando se realiza un trabajo. Se recomienda el uso del editor TeXstudio y un Sistema Operativo basado en GNU/Linux. La plantilla puede usarse tanto para trabajos en Español como en Inglés (en cuyo caso hay que descomentar la variable \emph{spanishfalse}).
%
%Esta plantilla ha sido creada por \href{https://github.com/JoseAngelMartinB}{José Ángel Martín Baos}. Está basada en la plantilla de la Universidad de Cape Town: \url{https://www.overleaf.com/latex/templates/uct-report-template/grctkzjtrqrm#.WVTJsXXyiV4} además de en los contenidos del curso “\LaTeX{} esencial para preparación de Trabajo Fin de Grado, Tesis y otros documentos académicos” impartido por el profesor Jesús Salido: \url{http://visilab.etsii.uclm.es/?page_id=1468}.
%
%\begin{figure}[H]
%	\centering
%	\includegraphics[angle=0]{licencia}
%\end{figure}
%Esta obra está bajo una licencia GNU General Public License Versión 3.
%\url{https://github.com/JoseAngelMartinB/PlantillaTrabajosLaTeX/blob/master/LICENSE.txt}
%
%
%% Sección: Instalación de LaTeX y dependencias de la plantilla
%\section{Instalación de \LaTeX{} y dependencias de la plantilla}
%Hay distintas formas de instalar todos los componentes necesarios, pero se recomienda seguir los pasos aquí dados usando un sistema operativo basado en Debian como Ubuntu:
%
%Una forma de instalar \LaTeX{} consiste en instalar la plantilla \emph{esi-tfg}, que proporciona una plantilla para el desarrollo del Trabajo Fin de Grado de la ESI, pero además instala todos los componentes necesarios para que \LaTeX{} funcione correctamente. Toda la documentación se encuentra en: \url{https://bitbucket.org/arco_group/esi-tfg}. Otra forma sería instalando \LaTeX{} directamente mediante:
%\begin{listing}[style=consola, numbers=none]
%$ sudo apt-get install texlive-full
%\end{listing} %$
%
%El siguiente paso sería instalar un editor de \LaTeX{}, recomendamos el uso de TeXstudio que puede obtenerse desde: \url{http://www.texstudio.org/}.
%
%El último paso sería instalar algunas dependencias necesarias para el funcionamiento de esta plantilla:
%\begin{listing}[style=consola, numbers=none]
%$ sudo apt-get install texlive-science
%\end{listing} %$
%
%
%% Sección: Ejemplos de tipografía y organización del documento
%\section{Ejemplos de tipografía y organización del documento}
%En esta sección se explica brevemente como usar algunas características básicas de \LaTeX como pueden ser \textbf{texto en negrita}, \emph{enfatizado}, \textit{cursiva}, \underline{subrayado}, \textsc{o versalitas}.
%
%\subsection{Subsecciones}
%Además se pueden crear distintas subsecciones y éstas a su vez incluir subsubsecciones:
%\subsubsection{Subsubsección 1}
%
%
%% Sección: Fórmulas en LaTeX
%\section{Fórmulas en \LaTeX{}}
%\LaTeX{} puede utilizarse para incorporar fórmulas matemáticas:
%$\sum_{n=1}^\infty\frac{1}{n^2}=\frac{\pi^2}{6}$
%
%Aunque también pueden expresarse así:
%\begin{equation} \label{eq:equation1}
%\sum_{n=1}^\infty\frac{1}{n^2}=\frac{\pi^2}{6}
%\end{equation}
%ó asi:
%\begin{equation*}
%\sum_{n=1}^\infty\frac{1}{n^2}=\frac{\pi^2}{6}
%\end{equation*}
%
%Y posteriormente hacer referencia a dicha ecuación: Ecuación (\ref{eq:equation1}).
%
%Se recomienda el uso de la herramienta Daumc Equation Editor (disponible en: \url{http://s1.daumcdn.net/editor/fp/service_nc/pencil/Pencil_chromestore.html}) para escribir ecuaciones en \LaTeX{} de manera trivial.
%
%
%% Sección: Ejemplo de códigos e imágenes
%\section{Ejemplo de códigos e imágenes}
%Podemos insertar comandos de consola de la siguiente forma:  \texttt{uname -a} , o mediante:
%\begin{listing}[style=consola, numbers=none]
%$ uname -a
%Linux droideka 4.4.0-66-generic #87-Ubuntu SMP Fri Mar 3 15:29:05 UTC 2017 x86_64 x86_64 x86_64 GNU/Linux
%\end{listing} %$
%
%También se pueden insertar códigos (listados) de la siguiente forma:
%\lstinputlisting[style=C,
%caption={Ejemplo de un listado de código},
%label={lst:HolaMundo},
%firstline=1]{code/hola_mundo.c}
%
%Podemos hacer referencia en cualquier momento a Listado \ref{lst:HolaMundo}
%
%Además de códigos, podemos insertar también pseudocódigos como se puede ver en Algoritmo \ref{alg:fox}. Para ello se utiliza el paquete algorithm2e.sty cuyo manual está disponible en: \url{http://osl.ugr.es/CTAN/macros/latex/contrib/algorithm2e/doc/algorithm2e.pdf}. Este paquete puede instalarse en cualquier distribución basada en Debian mediante el comando:
%\begin{listing}[style=consola, numbers=none]
%$ sudo apt-get install texlive-science
%\end{listing} %$
%
%\IncMargin{1em}
%\begin{algorithm}
%	\SetKwInOut{Input}{Datos}\SetKwInOut{Output}{Resultado}
%	\LinesNumbered
%	\SetAlgoLined
%
%	\Input{Matrices A y B, lado de la grilla de procesos (m), fila del proceso (i), columna del proceso (j)}
%	\Output{Matriz C}
%
%	\For{k = 0 to m-1}{
%		\eIf{j == ((i + k) mod m)}{
%			Broadcast $A_{ij}$ a los procesos de la misma fila (i)\;
%		}{
%			Receive $A_{ip}$ de los procesos de la misma fila\;
%		}
%		$C_{ij}$ += $A_{ip}$ * $B_{ij}$\;
%		\tcc{Mandar $B_{ij}$ al proceso de la fila superior y recibirlo del proceso de fila inferior}
%		Send $B_{ij}$ al proceso $t_{(i-1) j}$\;
%		Receive $B_{ij}$ del proceso $t_{(i+1) j}$\;
%	}
%
%	\caption{Algoritmo de Fox}\label{alg:fox}
%\end{algorithm}\DecMargin{1em}
%
%Para insertar imágenes (figuras) podemos usar los siguientes comandos, ya sea una imagen individual o dos subimágenes. También podemos hacer referencia a dichas imágenes: Figura \ref{fig:PlazaCR} o Figura \ref{fig:ESI1}.
%
%% Una imagen
%\begin{figure}[H] %H --> Obliga a que la imagen se coloque en el lugar exacto en el texto
%	\centering
%	\includegraphics[width=1\linewidth,angle=0]{PlazaCR}
%	\caption{Plaza de Ciudad Real}
%	\label{fig:PlazaCR}
%\end{figure}
%
%% Dos subimagenes
%\begin{figure}[htb]
%	\centering
%	\subfigure[Imagen de la fachada de la ESI]{
%		\includegraphics[width=7cm]{ESI2}
%		\label{fig:ESI1}
%	}
%	\subfigure[Imagen de la ESI]{
%		\includegraphics[width=7cm]{ESI1}
%		\label{fig:ESI2}
%	}
%	\caption{Imágenes que muestran la ESI}
%	\label{fig:ESI}
%\end{figure}
%
%Por último, para recrear la pulsación de un botón podemos usar: \tecla{Ctrl+C}
%
%
%% Sección: Ejemplo de listados
%\section{Ejemplo de listados}
%
%El primer ejemplo consiste en un listado normal:
%\begin{itemize}
%	\item Item 1
%	\item Item 2
%	\item[\ding{108}] Se puede cambiar de icono usando ding. Consultar ChuLaTeX \cite{Salido2011}
%	\item[*] Item 4
%\end{itemize}
%
%El segundo de un listado numerado:
%\begin{enumerate}
%	\item Item 1
%	\item Item 2
%	\item Item 3
%	\item Item 4
%\end{enumerate}
%
%En este ejemplo vamos a mostrar el uso de un listado compacto:
%
%\begin{compactitem}
%	\item Item 1
%	\item Item 2
%	\item Item 3
%	\item Item 4
%\end{compactitem}
%
%
%% Sección: Tablas
%\section{Tablas}
%Es posible introducir tablas en \LaTeX{}. Aquí se puede ver un pequeño ejemplo:
%
%\begin{table}[htbp]
%	\begin{center}
%		\begin{tabular}{|l|l|}
%			\hline
%			Número de procesos & Tiempo (segundos) \\
%			\hline \hline
%			1 & 555,273043 \\ \hline
%			4 & 278,392832 \\ \hline
%			9 & 260,050554 \\ \hline
%			16 & 251,819869 \\ \hline
%			25 & 236,560818 \\ \hline
%		\end{tabular}
%		\caption{Resultado de la ejecución para matrices de orden N = 3600.}
%		\label{tabla:Tiempos3600}
%	\end{center}
%\end{table}
%
%
%% Sección: Referencias
%\section{Añadir referencias}
%Para el manejo de las referencias se recomienda la instalación de JabRef disponible en: \url{http://www.jabref.org/}. Mediante esta aplicación abrimos el archivo \emph{biblist.bib}. Mediante la aplicación introducimos las distintas referencias que queramos y a cada una le asignamos una bibtexkey que sea significativa para nosotros y no se repita.
%
%Por último podemos citar libros o artículos mediante el comando cite. \cite{Kottwitz2011} \cite{Martin2017} \cite{Salido2011}. En este comando ponemos la bibtexkey que indica la referencia que queremos introducir.
%
%
%% Sección: Mejoras y sugerencias
%\section{Mejoras y sugerencias}
%Puedes ayudar al desarrollo de esta plantilla aportando tus ideas, mejoras o sugerencias. Para ello puedes ponerte en contacto conmigo mediante la dirección de correo electrónico: \emph{joseangelmartinb@gmail.com}
%
%

%%%%%%%%%%%%%%%%%%%%%%%%%%%%%%%%%%%%%%%%%%%%%%%%%%%%%%%%%%%%%%%%%%%%%%%%%%%%%%%%
%%%%%%%%% 						BIBLIOGRAFIA 						   %%%%%%%%%
%%%%%%%%%%%%%%%%%%%%%%%%%%%%%%%%%%%%%%%%%%%%%%%%%%%%%%%%%%%%%%%%%%%%%%%%%%%%%%%%
\newpage
\bibliography{biblist}
\bibliographystyle{plain}
%\nocite{*} Permite citar todas las referencias en el archivo .bib

% Añadir la bibliografía al Índice de contenidos
\ifspanish
	\addcontentsline{toc}{section}{Referencias}
\else
	\addcontentsline{toc}{section}{References}
\fi



%%%%%%%%%%%%%%%%%%%%%%%%%%%%%%%%%%%%%%%%%%%%%%%%%%%%%%%%%%%%%%%%%%%%%%%%%%%%%%%%
%%%%%%%%% 					Atribución - ¡No Eliminar!				   %%%%%%%%%
%%%%%%%%%%%%%%%%%%%%%%%%%%%%%%%%%%%%%%%%%%%%%%%%%%%%%%%%%%%%%%%%%%%%%%%%%%%%%%%%
%\null\vfill
%\begin{center}
%\ifspanish
%	Este documento ha sido generado con \LaTeX{} utilizando la plantilla\\
%	desarrollada por \textsc{José Ángel Martín Baos} y disponible en\\
%	\url{https://github.com/JoseAngelMartinB/PlantillaTrabajosLaTeX}
%\else
%	This document was generated using \LaTeX{} and the template\\
%	developed by \textsc{José Ángel Martín Baos} which is available in\\
%	\url{https://github.com/JoseAngelMartinB/PlantillaTrabajosLaTeX}
%\fi
%\end{center}

\end{document}
